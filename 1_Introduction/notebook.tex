
% Default to the notebook output style

    


% Inherit from the specified cell style.




    
\documentclass[11pt]{article}

    
    
    \usepackage[T1]{fontenc}
    % Nicer default font (+ math font) than Computer Modern for most use cases
    \usepackage{mathpazo}

    % Basic figure setup, for now with no caption control since it's done
    % automatically by Pandoc (which extracts ![](path) syntax from Markdown).
    \usepackage{graphicx}
    % We will generate all images so they have a width \maxwidth. This means
    % that they will get their normal width if they fit onto the page, but
    % are scaled down if they would overflow the margins.
    \makeatletter
    \def\maxwidth{\ifdim\Gin@nat@width>\linewidth\linewidth
    \else\Gin@nat@width\fi}
    \makeatother
    \let\Oldincludegraphics\includegraphics
    % Set max figure width to be 80% of text width, for now hardcoded.
    \renewcommand{\includegraphics}[1]{\Oldincludegraphics[width=.8\maxwidth]{#1}}
    % Ensure that by default, figures have no caption (until we provide a
    % proper Figure object with a Caption API and a way to capture that
    % in the conversion process - todo).
    \usepackage{caption}
    \DeclareCaptionLabelFormat{nolabel}{}
    \captionsetup{labelformat=nolabel}

    \usepackage{adjustbox} % Used to constrain images to a maximum size 
    \usepackage{xcolor} % Allow colors to be defined
    \usepackage{enumerate} % Needed for markdown enumerations to work
    \usepackage{geometry} % Used to adjust the document margins
    \usepackage{amsmath} % Equations
    \usepackage{amssymb} % Equations
    \usepackage{textcomp} % defines textquotesingle
    % Hack from http://tex.stackexchange.com/a/47451/13684:
    \AtBeginDocument{%
        \def\PYZsq{\textquotesingle}% Upright quotes in Pygmentized code
    }
    \usepackage{upquote} % Upright quotes for verbatim code
    \usepackage{eurosym} % defines \euro
    \usepackage[mathletters]{ucs} % Extended unicode (utf-8) support
    \usepackage[utf8x]{inputenc} % Allow utf-8 characters in the tex document
    \usepackage{fancyvrb} % verbatim replacement that allows latex
    \usepackage{grffile} % extends the file name processing of package graphics 
                         % to support a larger range 
    % The hyperref package gives us a pdf with properly built
    % internal navigation ('pdf bookmarks' for the table of contents,
    % internal cross-reference links, web links for URLs, etc.)
    \usepackage{hyperref}
    \usepackage{longtable} % longtable support required by pandoc >1.10
    \usepackage{booktabs}  % table support for pandoc > 1.12.2
    \usepackage[inline]{enumitem} % IRkernel/repr support (it uses the enumerate* environment)
    \usepackage[normalem]{ulem} % ulem is needed to support strikethroughs (\sout)
                                % normalem makes italics be italics, not underlines
    

    
    
    % Colors for the hyperref package
    \definecolor{urlcolor}{rgb}{0,.145,.698}
    \definecolor{linkcolor}{rgb}{.71,0.21,0.01}
    \definecolor{citecolor}{rgb}{.12,.54,.11}

    % ANSI colors
    \definecolor{ansi-black}{HTML}{3E424D}
    \definecolor{ansi-black-intense}{HTML}{282C36}
    \definecolor{ansi-red}{HTML}{E75C58}
    \definecolor{ansi-red-intense}{HTML}{B22B31}
    \definecolor{ansi-green}{HTML}{00A250}
    \definecolor{ansi-green-intense}{HTML}{007427}
    \definecolor{ansi-yellow}{HTML}{DDB62B}
    \definecolor{ansi-yellow-intense}{HTML}{B27D12}
    \definecolor{ansi-blue}{HTML}{208FFB}
    \definecolor{ansi-blue-intense}{HTML}{0065CA}
    \definecolor{ansi-magenta}{HTML}{D160C4}
    \definecolor{ansi-magenta-intense}{HTML}{A03196}
    \definecolor{ansi-cyan}{HTML}{60C6C8}
    \definecolor{ansi-cyan-intense}{HTML}{258F8F}
    \definecolor{ansi-white}{HTML}{C5C1B4}
    \definecolor{ansi-white-intense}{HTML}{A1A6B2}

    % commands and environments needed by pandoc snippets
    % extracted from the output of `pandoc -s`
    \providecommand{\tightlist}{%
      \setlength{\itemsep}{0pt}\setlength{\parskip}{0pt}}
    \DefineVerbatimEnvironment{Highlighting}{Verbatim}{commandchars=\\\{\}}
    % Add ',fontsize=\small' for more characters per line
    \newenvironment{Shaded}{}{}
    \newcommand{\KeywordTok}[1]{\textcolor[rgb]{0.00,0.44,0.13}{\textbf{{#1}}}}
    \newcommand{\DataTypeTok}[1]{\textcolor[rgb]{0.56,0.13,0.00}{{#1}}}
    \newcommand{\DecValTok}[1]{\textcolor[rgb]{0.25,0.63,0.44}{{#1}}}
    \newcommand{\BaseNTok}[1]{\textcolor[rgb]{0.25,0.63,0.44}{{#1}}}
    \newcommand{\FloatTok}[1]{\textcolor[rgb]{0.25,0.63,0.44}{{#1}}}
    \newcommand{\CharTok}[1]{\textcolor[rgb]{0.25,0.44,0.63}{{#1}}}
    \newcommand{\StringTok}[1]{\textcolor[rgb]{0.25,0.44,0.63}{{#1}}}
    \newcommand{\CommentTok}[1]{\textcolor[rgb]{0.38,0.63,0.69}{\textit{{#1}}}}
    \newcommand{\OtherTok}[1]{\textcolor[rgb]{0.00,0.44,0.13}{{#1}}}
    \newcommand{\AlertTok}[1]{\textcolor[rgb]{1.00,0.00,0.00}{\textbf{{#1}}}}
    \newcommand{\FunctionTok}[1]{\textcolor[rgb]{0.02,0.16,0.49}{{#1}}}
    \newcommand{\RegionMarkerTok}[1]{{#1}}
    \newcommand{\ErrorTok}[1]{\textcolor[rgb]{1.00,0.00,0.00}{\textbf{{#1}}}}
    \newcommand{\NormalTok}[1]{{#1}}
    
    % Additional commands for more recent versions of Pandoc
    \newcommand{\ConstantTok}[1]{\textcolor[rgb]{0.53,0.00,0.00}{{#1}}}
    \newcommand{\SpecialCharTok}[1]{\textcolor[rgb]{0.25,0.44,0.63}{{#1}}}
    \newcommand{\VerbatimStringTok}[1]{\textcolor[rgb]{0.25,0.44,0.63}{{#1}}}
    \newcommand{\SpecialStringTok}[1]{\textcolor[rgb]{0.73,0.40,0.53}{{#1}}}
    \newcommand{\ImportTok}[1]{{#1}}
    \newcommand{\DocumentationTok}[1]{\textcolor[rgb]{0.73,0.13,0.13}{\textit{{#1}}}}
    \newcommand{\AnnotationTok}[1]{\textcolor[rgb]{0.38,0.63,0.69}{\textbf{\textit{{#1}}}}}
    \newcommand{\CommentVarTok}[1]{\textcolor[rgb]{0.38,0.63,0.69}{\textbf{\textit{{#1}}}}}
    \newcommand{\VariableTok}[1]{\textcolor[rgb]{0.10,0.09,0.49}{{#1}}}
    \newcommand{\ControlFlowTok}[1]{\textcolor[rgb]{0.00,0.44,0.13}{\textbf{{#1}}}}
    \newcommand{\OperatorTok}[1]{\textcolor[rgb]{0.40,0.40,0.40}{{#1}}}
    \newcommand{\BuiltInTok}[1]{{#1}}
    \newcommand{\ExtensionTok}[1]{{#1}}
    \newcommand{\PreprocessorTok}[1]{\textcolor[rgb]{0.74,0.48,0.00}{{#1}}}
    \newcommand{\AttributeTok}[1]{\textcolor[rgb]{0.49,0.56,0.16}{{#1}}}
    \newcommand{\InformationTok}[1]{\textcolor[rgb]{0.38,0.63,0.69}{\textbf{\textit{{#1}}}}}
    \newcommand{\WarningTok}[1]{\textcolor[rgb]{0.38,0.63,0.69}{\textbf{\textit{{#1}}}}}
    
    
    % Define a nice break command that doesn't care if a line doesn't already
    % exist.
    \def\br{\hspace*{\fill} \\* }
    % Math Jax compatability definitions
    \def\gt{>}
    \def\lt{<}
    % Document parameters
    \title{2020\_DPO\_0\_git}
    
    
    

    % Pygments definitions
    
\makeatletter
\def\PY@reset{\let\PY@it=\relax \let\PY@bf=\relax%
    \let\PY@ul=\relax \let\PY@tc=\relax%
    \let\PY@bc=\relax \let\PY@ff=\relax}
\def\PY@tok#1{\csname PY@tok@#1\endcsname}
\def\PY@toks#1+{\ifx\relax#1\empty\else%
    \PY@tok{#1}\expandafter\PY@toks\fi}
\def\PY@do#1{\PY@bc{\PY@tc{\PY@ul{%
    \PY@it{\PY@bf{\PY@ff{#1}}}}}}}
\def\PY#1#2{\PY@reset\PY@toks#1+\relax+\PY@do{#2}}

\expandafter\def\csname PY@tok@w\endcsname{\def\PY@tc##1{\textcolor[rgb]{0.73,0.73,0.73}{##1}}}
\expandafter\def\csname PY@tok@c\endcsname{\let\PY@it=\textit\def\PY@tc##1{\textcolor[rgb]{0.25,0.50,0.50}{##1}}}
\expandafter\def\csname PY@tok@cp\endcsname{\def\PY@tc##1{\textcolor[rgb]{0.74,0.48,0.00}{##1}}}
\expandafter\def\csname PY@tok@k\endcsname{\let\PY@bf=\textbf\def\PY@tc##1{\textcolor[rgb]{0.00,0.50,0.00}{##1}}}
\expandafter\def\csname PY@tok@kp\endcsname{\def\PY@tc##1{\textcolor[rgb]{0.00,0.50,0.00}{##1}}}
\expandafter\def\csname PY@tok@kt\endcsname{\def\PY@tc##1{\textcolor[rgb]{0.69,0.00,0.25}{##1}}}
\expandafter\def\csname PY@tok@o\endcsname{\def\PY@tc##1{\textcolor[rgb]{0.40,0.40,0.40}{##1}}}
\expandafter\def\csname PY@tok@ow\endcsname{\let\PY@bf=\textbf\def\PY@tc##1{\textcolor[rgb]{0.67,0.13,1.00}{##1}}}
\expandafter\def\csname PY@tok@nb\endcsname{\def\PY@tc##1{\textcolor[rgb]{0.00,0.50,0.00}{##1}}}
\expandafter\def\csname PY@tok@nf\endcsname{\def\PY@tc##1{\textcolor[rgb]{0.00,0.00,1.00}{##1}}}
\expandafter\def\csname PY@tok@nc\endcsname{\let\PY@bf=\textbf\def\PY@tc##1{\textcolor[rgb]{0.00,0.00,1.00}{##1}}}
\expandafter\def\csname PY@tok@nn\endcsname{\let\PY@bf=\textbf\def\PY@tc##1{\textcolor[rgb]{0.00,0.00,1.00}{##1}}}
\expandafter\def\csname PY@tok@ne\endcsname{\let\PY@bf=\textbf\def\PY@tc##1{\textcolor[rgb]{0.82,0.25,0.23}{##1}}}
\expandafter\def\csname PY@tok@nv\endcsname{\def\PY@tc##1{\textcolor[rgb]{0.10,0.09,0.49}{##1}}}
\expandafter\def\csname PY@tok@no\endcsname{\def\PY@tc##1{\textcolor[rgb]{0.53,0.00,0.00}{##1}}}
\expandafter\def\csname PY@tok@nl\endcsname{\def\PY@tc##1{\textcolor[rgb]{0.63,0.63,0.00}{##1}}}
\expandafter\def\csname PY@tok@ni\endcsname{\let\PY@bf=\textbf\def\PY@tc##1{\textcolor[rgb]{0.60,0.60,0.60}{##1}}}
\expandafter\def\csname PY@tok@na\endcsname{\def\PY@tc##1{\textcolor[rgb]{0.49,0.56,0.16}{##1}}}
\expandafter\def\csname PY@tok@nt\endcsname{\let\PY@bf=\textbf\def\PY@tc##1{\textcolor[rgb]{0.00,0.50,0.00}{##1}}}
\expandafter\def\csname PY@tok@nd\endcsname{\def\PY@tc##1{\textcolor[rgb]{0.67,0.13,1.00}{##1}}}
\expandafter\def\csname PY@tok@s\endcsname{\def\PY@tc##1{\textcolor[rgb]{0.73,0.13,0.13}{##1}}}
\expandafter\def\csname PY@tok@sd\endcsname{\let\PY@it=\textit\def\PY@tc##1{\textcolor[rgb]{0.73,0.13,0.13}{##1}}}
\expandafter\def\csname PY@tok@si\endcsname{\let\PY@bf=\textbf\def\PY@tc##1{\textcolor[rgb]{0.73,0.40,0.53}{##1}}}
\expandafter\def\csname PY@tok@se\endcsname{\let\PY@bf=\textbf\def\PY@tc##1{\textcolor[rgb]{0.73,0.40,0.13}{##1}}}
\expandafter\def\csname PY@tok@sr\endcsname{\def\PY@tc##1{\textcolor[rgb]{0.73,0.40,0.53}{##1}}}
\expandafter\def\csname PY@tok@ss\endcsname{\def\PY@tc##1{\textcolor[rgb]{0.10,0.09,0.49}{##1}}}
\expandafter\def\csname PY@tok@sx\endcsname{\def\PY@tc##1{\textcolor[rgb]{0.00,0.50,0.00}{##1}}}
\expandafter\def\csname PY@tok@m\endcsname{\def\PY@tc##1{\textcolor[rgb]{0.40,0.40,0.40}{##1}}}
\expandafter\def\csname PY@tok@gh\endcsname{\let\PY@bf=\textbf\def\PY@tc##1{\textcolor[rgb]{0.00,0.00,0.50}{##1}}}
\expandafter\def\csname PY@tok@gu\endcsname{\let\PY@bf=\textbf\def\PY@tc##1{\textcolor[rgb]{0.50,0.00,0.50}{##1}}}
\expandafter\def\csname PY@tok@gd\endcsname{\def\PY@tc##1{\textcolor[rgb]{0.63,0.00,0.00}{##1}}}
\expandafter\def\csname PY@tok@gi\endcsname{\def\PY@tc##1{\textcolor[rgb]{0.00,0.63,0.00}{##1}}}
\expandafter\def\csname PY@tok@gr\endcsname{\def\PY@tc##1{\textcolor[rgb]{1.00,0.00,0.00}{##1}}}
\expandafter\def\csname PY@tok@ge\endcsname{\let\PY@it=\textit}
\expandafter\def\csname PY@tok@gs\endcsname{\let\PY@bf=\textbf}
\expandafter\def\csname PY@tok@gp\endcsname{\let\PY@bf=\textbf\def\PY@tc##1{\textcolor[rgb]{0.00,0.00,0.50}{##1}}}
\expandafter\def\csname PY@tok@go\endcsname{\def\PY@tc##1{\textcolor[rgb]{0.53,0.53,0.53}{##1}}}
\expandafter\def\csname PY@tok@gt\endcsname{\def\PY@tc##1{\textcolor[rgb]{0.00,0.27,0.87}{##1}}}
\expandafter\def\csname PY@tok@err\endcsname{\def\PY@bc##1{\setlength{\fboxsep}{0pt}\fcolorbox[rgb]{1.00,0.00,0.00}{1,1,1}{\strut ##1}}}
\expandafter\def\csname PY@tok@kc\endcsname{\let\PY@bf=\textbf\def\PY@tc##1{\textcolor[rgb]{0.00,0.50,0.00}{##1}}}
\expandafter\def\csname PY@tok@kd\endcsname{\let\PY@bf=\textbf\def\PY@tc##1{\textcolor[rgb]{0.00,0.50,0.00}{##1}}}
\expandafter\def\csname PY@tok@kn\endcsname{\let\PY@bf=\textbf\def\PY@tc##1{\textcolor[rgb]{0.00,0.50,0.00}{##1}}}
\expandafter\def\csname PY@tok@kr\endcsname{\let\PY@bf=\textbf\def\PY@tc##1{\textcolor[rgb]{0.00,0.50,0.00}{##1}}}
\expandafter\def\csname PY@tok@bp\endcsname{\def\PY@tc##1{\textcolor[rgb]{0.00,0.50,0.00}{##1}}}
\expandafter\def\csname PY@tok@fm\endcsname{\def\PY@tc##1{\textcolor[rgb]{0.00,0.00,1.00}{##1}}}
\expandafter\def\csname PY@tok@vc\endcsname{\def\PY@tc##1{\textcolor[rgb]{0.10,0.09,0.49}{##1}}}
\expandafter\def\csname PY@tok@vg\endcsname{\def\PY@tc##1{\textcolor[rgb]{0.10,0.09,0.49}{##1}}}
\expandafter\def\csname PY@tok@vi\endcsname{\def\PY@tc##1{\textcolor[rgb]{0.10,0.09,0.49}{##1}}}
\expandafter\def\csname PY@tok@vm\endcsname{\def\PY@tc##1{\textcolor[rgb]{0.10,0.09,0.49}{##1}}}
\expandafter\def\csname PY@tok@sa\endcsname{\def\PY@tc##1{\textcolor[rgb]{0.73,0.13,0.13}{##1}}}
\expandafter\def\csname PY@tok@sb\endcsname{\def\PY@tc##1{\textcolor[rgb]{0.73,0.13,0.13}{##1}}}
\expandafter\def\csname PY@tok@sc\endcsname{\def\PY@tc##1{\textcolor[rgb]{0.73,0.13,0.13}{##1}}}
\expandafter\def\csname PY@tok@dl\endcsname{\def\PY@tc##1{\textcolor[rgb]{0.73,0.13,0.13}{##1}}}
\expandafter\def\csname PY@tok@s2\endcsname{\def\PY@tc##1{\textcolor[rgb]{0.73,0.13,0.13}{##1}}}
\expandafter\def\csname PY@tok@sh\endcsname{\def\PY@tc##1{\textcolor[rgb]{0.73,0.13,0.13}{##1}}}
\expandafter\def\csname PY@tok@s1\endcsname{\def\PY@tc##1{\textcolor[rgb]{0.73,0.13,0.13}{##1}}}
\expandafter\def\csname PY@tok@mb\endcsname{\def\PY@tc##1{\textcolor[rgb]{0.40,0.40,0.40}{##1}}}
\expandafter\def\csname PY@tok@mf\endcsname{\def\PY@tc##1{\textcolor[rgb]{0.40,0.40,0.40}{##1}}}
\expandafter\def\csname PY@tok@mh\endcsname{\def\PY@tc##1{\textcolor[rgb]{0.40,0.40,0.40}{##1}}}
\expandafter\def\csname PY@tok@mi\endcsname{\def\PY@tc##1{\textcolor[rgb]{0.40,0.40,0.40}{##1}}}
\expandafter\def\csname PY@tok@il\endcsname{\def\PY@tc##1{\textcolor[rgb]{0.40,0.40,0.40}{##1}}}
\expandafter\def\csname PY@tok@mo\endcsname{\def\PY@tc##1{\textcolor[rgb]{0.40,0.40,0.40}{##1}}}
\expandafter\def\csname PY@tok@ch\endcsname{\let\PY@it=\textit\def\PY@tc##1{\textcolor[rgb]{0.25,0.50,0.50}{##1}}}
\expandafter\def\csname PY@tok@cm\endcsname{\let\PY@it=\textit\def\PY@tc##1{\textcolor[rgb]{0.25,0.50,0.50}{##1}}}
\expandafter\def\csname PY@tok@cpf\endcsname{\let\PY@it=\textit\def\PY@tc##1{\textcolor[rgb]{0.25,0.50,0.50}{##1}}}
\expandafter\def\csname PY@tok@c1\endcsname{\let\PY@it=\textit\def\PY@tc##1{\textcolor[rgb]{0.25,0.50,0.50}{##1}}}
\expandafter\def\csname PY@tok@cs\endcsname{\let\PY@it=\textit\def\PY@tc##1{\textcolor[rgb]{0.25,0.50,0.50}{##1}}}

\def\PYZbs{\char`\\}
\def\PYZus{\char`\_}
\def\PYZob{\char`\{}
\def\PYZcb{\char`\}}
\def\PYZca{\char`\^}
\def\PYZam{\char`\&}
\def\PYZlt{\char`\<}
\def\PYZgt{\char`\>}
\def\PYZsh{\char`\#}
\def\PYZpc{\char`\%}
\def\PYZdl{\char`\$}
\def\PYZhy{\char`\-}
\def\PYZsq{\char`\'}
\def\PYZdq{\char`\"}
\def\PYZti{\char`\~}
% for compatibility with earlier versions
\def\PYZat{@}
\def\PYZlb{[}
\def\PYZrb{]}
\makeatother


    % Exact colors from NB
    \definecolor{incolor}{rgb}{0.0, 0.0, 0.5}
    \definecolor{outcolor}{rgb}{0.545, 0.0, 0.0}



    
    % Prevent overflowing lines due to hard-to-break entities
    \sloppy 
    % Setup hyperref package
    \hypersetup{
      breaklinks=true,  % so long urls are correctly broken across lines
      colorlinks=true,
      urlcolor=urlcolor,
      linkcolor=linkcolor,
      citecolor=citecolor,
      }
    % Slightly bigger margins than the latex defaults
    
    \geometry{verbose,tmargin=1in,bmargin=1in,lmargin=1in,rmargin=1in}
    
    

    \begin{document}
    
    
    \maketitle
    
    

    
    Центр непрерывного образования

\section{Программа «Python для автоматизации и анализа
данных»}\label{ux43fux440ux43eux433ux440ux430ux43cux43cux430-python-ux434ux43bux44f-ux430ux432ux442ux43eux43cux430ux442ux438ux437ux430ux446ux438ux438-ux438-ux430ux43dux430ux43bux438ux437ux430-ux434ux430ux43dux43dux44bux445}

Неделя 1 - 1

\emph{Рогович Татьяна, НИУ ВШЭ}

На основе
\href{https://ancatmara.gitbooks.io/digital-literacy/chapter1.html}{статьи}.

    \subsection{Что такое
Git?}\label{ux447ux442ux43e-ux442ux430ux43aux43eux435-git}

Git - \textbf{система управления версиями} (\emph{version control
system, VCS}), созданная программистом Линусом Торвальдсом для
управления разработкой ядра Linux в 2005 году. Хорошо, а что это
всё-таки значит?

Представьте, что вы с коллегами вместе пишете научную статью. У вас на
компьютере есть папка, где лежат текстовые документы, картинки, графики
и прочие нужные файлы; то же самое есть и у ваших коллег. Когда кто-то
из вас изменяет, добавляет или удаляет файлы, остальные этих изменений
не видят. Вы пишете друг другу об изменениях, пересылаете обновленные
версии файлов, но в процессе работы непременно возникает путаница: какая
версия текста - последняя? Куда и когда исчезла пара абзацев? Кто внес
те или иные правки? Избежать таких проблем и помогают системы контроля
версий. Устроено это так:

\begin{itemize}
\tightlist
\item
  Ваша папка на компьютере - это не просто папка, а локальный
  репозиторий
\item
  Она является копией удаленного репозитория, который лежит на
  веб-хостинге (например, GitHub или BitBucket)
\item
  Eсли вы работаете над проектом с коллегами, то своя локальная копия
  есть у каждого
\item
  Kогда вы внесли некоторое количество изменений, вы можете их
  сохранить, и это дествие запишется в журнал; это называется commit
\item
  После этого можно отправить изменения в удаленный репозиторий; это
  называется push
\item
  Актуальная версия проекта, учитывающая последние изменения всех
  участников, будет храниться в удаленном репозитории
\item
  Если вы увидели, что ваши коллеги запушили в удаленный репозиторий
  что-то новенькое, то можно (и нужно!) скопировать это себе на
  компьютер; это называется pull
\end{itemize}

Чем-то похоже на Dropbox, GoogleDrive и прочие облачные хранилища,
правда? Только в данном случае ваши файлы синхронизируются не
автоматически, а по команде, и возможностей управления ими гораздо
больше.

Понятно, что для совместной работы над текстом научной статьи вполне
хватит и GoogleDocs, но вот если, например, вы хотите опубликовать
результаты исследования в интернете и сделать для этого собственный
сайт, то без VCS обойтись сложно. И ещё раз, системы контроля версий
хороши тем, что

\begin{itemize}
\tightlist
\item
  Позволяют работать над проектом в команде;
\item
  Вы видите, кем и когда были внесены те или иные изменения;
\item
  Их всегда можно откатить назад;
\item
  Вы не потеряете проделанную работу, даже если что-то удалите на своем
  компьютере;
\item
  Ваши наработки полностью открыты для других (а это доступность знаний
  и ускорение развития технологий, ура!);
\item
  GitHub позволяет не только хранить и просматривать файлы проекта, но и
  публиковать веб-сайты, документацию и т.п.
\end{itemize}

Существует много систем управления версиями, но мы будем пользоваться
самой распространенной - \textbf{git}. Также нам нужно как-то отдавать
гиту команды, и делать это можно двумя способами: с помощью командной
строки и через графический интерфейс (graphical user interface, GUI).
Графический интерфейс программы - это все те окошки с кнопочками,
которые мы привыкли видеть. Существует много графических интерфейсов для
гита, например:

\begin{itemize}
\tightlist
\item
  \href{https://git-scm.com/docs/git-gui}{Git GUI}
\item
  \href{https://desktop.github.com/}{GitHub Desktop}
\item
  \href{https://gitextensions.github.io/}{Git Extensions}
\item
  \href{https://www.sourcetreeapp.com/}{SourceTree}
\item
  \href{https://tortoisegit.org/}{TortoiseGit}
\end{itemize}

Мы будем пользоваться программой GitHub Desktop, которую можно скачать
\href{https://desktop.github.com/}{отсюда}. Если вы уже знакомы с гитом,
то можете выбрать любую программу или пользоваться командной строкой -
это не принципиально. Стоит отметить, что пользоваться командной строкой
гораздо сложнее чем графическим интерфейсом, поэтому она больше подходит
продвинутым пользователям, поэтому в данном блокноте, командная строка
рассматриваться не будет.

Давайте еще раз закрепим: * \textbf{git} - разновидность системы
контроля версий (самая популярная). Его можно скачать и установить,
далее использовать через командную строку * Можно использовать
графический интерфейс для работы с \textbf{git}. При этом скачивать и
устанавливать сам git отдельно не нужно, он обычно идет в комплекте с
графическим интерфейсом (но не во всех gui) * Репозиторий - это место
где мы храним наш код\проект и всю информацию по файлам, их изменения и
т.д. Репозиторий должен где-то хранится, чтобы у всех был доступ к нему
и они могли видеть изменения. Его можно хранить и на домашнем
компьютере, но не всегда удобно держать компьютер включенным целыми
сутками, поэтому используют хостинги для репозиториев. Одним из самых
известных является \textbf{\href{github.com}{Github}}

    \subsection{Github}\label{github}

\textbf{Github} - крупнейший веб-сервис для хостинга IT-проектов и их
совместной разработки. Веб-сервис основан на системе контроля версий Git
и разработан на Ruby on Rails и Erlang компанией GitHub, Inc. Так как мы
будем хранить на нем наши репозитории, поэтому мы и выбрали
\textbf{Github Desktop}, т.к. он разрабатывается специально для
максимальной интеграции и упрощения с Github.

Для работы нам нужен аккаунт. Чтобы зарегистрироваться, идём
\href{https://github.com/join?source=header-home}{сюда}, выбираем имя
пользователя и пароль, и готово! Теперь у вас есть собственная
страничка: https://github.com/username, где username - имя пользователя
которое вы указали.

После регистрации вы попадете на приветственную страницу, где сначала
нужно, ничего не меняя, нажать зеленую кнопку \textbf{Continue}, а потом
\textbf{Skip this step} (но если не лень, можно заполнить опросник и
нажать \textbf{Submit}).

Далее подтвердите свой аккаунт на указанной ранее почте и все, вы готовы
к работе.

    \subsection{Как скачать материалы
курса}\label{ux43aux430ux43a-ux441ux43aux430ux447ux430ux442ux44c-ux43cux430ux442ux435ux440ux438ux430ux43bux44b-ux43aux443ux440ux441ux430}

На самом деле для простого скачивания материалов курса, не нужен аккаунт
в Github. Все можно сделать через зеленую кнопку \textbf{Clone or
download}

Нам нужно нажать \textbf{Download ZIP} и тогда архив со всеми файлами
репозитория будет скачан на ваш компьютер

Отдельные файлы, например, блокноты .ipynb, можно скачать индивидуально.
Находим необходимый файл и кликаем на него

Нам откроется сам файл и его содержимое. Нажимаем на кнопку \textbf{RAW}

Теперь перед нам просто \emph{текст} файла и его можно сохранить. Во
многих браузерах на основе Chromium, есть функция \textbf{сохранить}
(Google Chrome, Yandex Browser и т.д.). Ее мы и нажимаем.

Если кнопки нет, то мы можем скопировать весь текст (CTR + A) и вставить
его в пустой тесктовый файл и затем сохранить его с расширением
\textbf{.ipynb}. Также, файл скачанный из браузера сохраняется с
расширением \textbf{.txt}, поэтому его нужно стереть и оставить
\textbf{.ipynb}.

На \textbf{Windows} это можно сделать, нажав правой кнопкой и затем
переименовать, или нажав \textbf{F2} (Файл должен быть выделен).
Соглашаемся с диалоговым окном

На \textbf{MacOS} это можно сделать, нажав правой кнопкой и затем
переименовать, или нажав \textbf{Enter/Return} (файл должен быть
выделен). Соглашаемся с диалоговым окном

Есть еще третий способ скачивания ресурсов это \textbf{Clone} или
клонирование, но к нему мы перейдем после создания репозитория, который
и склонируем.

    \subsection{Создание
репозитория}\label{ux441ux43eux437ux434ux430ux43dux438ux435-ux440ux435ux43fux43eux437ux438ux442ux43eux440ux438ux44f}

Есть два основных способа создать репозиторий: * На сайте * Через GitHub
Desktop

Сначала создадим через сайт. Чтобы создать репозиторий, нажимаем кнопку
\textbf{Start a project} и выбираем название. Оно может быть любым, но
должно отражать суть того, что лежит внутри, например, "homeworks".
Впрочем, гитхаб предлагает более креативные варианты. Также в
специальном поле можно добавить описание. Для публичных репозиториев
хорощей практикой является заполнение всех полей, чтобы пользователи
могли срази понять о чем конкретно данный репозиторий.

    У нас есть выбор между \textbf{Public} и \textbf{Private}. Разница между
ними в том, что публичные репозиторий видно в поиске, в вашем профиле,
любой может просмотреть весь код и предложить свои исправления (pull
request). Приватный доступен только по ссылке, создатель репозитория сам
выбирает того кто видет репозиторий и кто может делать коммиты. На
обычном (бесплатном) аккаунте возможность создавать приватные
репозитории обычно ограничена. Для их создания можно попробовать
получить \textbf{PRO} доступ через
\href{https://education.github.com/}{Github Education} для учащихся с
почтой от учебного заведения. Мы будем использовать публичный
репозиторий.

Далее у нас есть возможность инициализировать репозиторий с файлом
Readme. В нем может быть отображена информация о репозитории, установке
файлов и т.д. Описание происходит в формет \textbf{Wiki Markdown}. Также
за этой галочкой скрывается команда \textbf{init}, которая превращает
пустую папку в git-проект.

Также стоит упомянуть про файл \textbf{.gitignore} и \textbf{license}.
Файл \textbf{.gitignore} нужен для того чтобы в репозиторий не попадали
разные временные файлы или сборки, например, при сборке проекта в Visual
Studio создается множество временных файлов, которые при каждом
изменении исходного кода программы, будут другими, поэтому для
репозитория это по факту \emph{мусор}. Поэтому в этом файле прописано,
что определенные папки и файлы не учитывать. При создании репозитория
можно выбрать уже заранее созданные файлы под язык программирования или
среду разработки. Также его можно прописать или дополнить и указать
какие файлы включить или убрать из репозитория. Файл \textbf{license}
указывает на то по какой лицензии распространяется код. Про каждую
лицензию можно почитать отдельно и в основном они отличаются тем что
можно делать с кодом: продавать, распространять, изменять и т.д. При
создании нашего репозитория можно либо выбрать эти файлы, либо оставить
их пустыми.

Далее нажимаем на зеленую кнопку \textbf{Create repository}. Вы увидите
список файлов в своем репозитории (пока это только автоматически
сгенерированный файл README с описанием проекта) и содержание README,
если он есть. Также файлы \textbf{.gitignore} и \textbf{license}, если
создавались. Ссылка на репозиторий будет выглядеть так:
https://github.com/username/your\_repo\_name.git

    \subsection{Клонируем
репозиторий}\label{ux43aux43bux43eux43dux438ux440ux443ux435ux43c-ux440ux435ux43fux43eux437ux438ux442ux43eux440ux438ux439}

Теперь нам нужно сделать локальную копию нашего удаленного репозитория.
Мы снова воспользуемся кнопкой \textbf{Clone or download}, но теперь
используем полную ссылку на репозиторий; эту ссылку нужно скопировать
(Если у вас окошко выглядит не так как на картинке, то нажмите в окне на
ссылку справа сверху Use HTTPS).

Для дальнеших шагов нам потребуется скачать и установить
\href{https://desktop.github.com/}{GitHub Desktop}. После установки и
первого запуска, возможно, потребуется войти в ваш аккаунт Github. Далее
выбираем \textbf{Clone repository} или через \textbf{File}, а затем уже
\textbf{Clone repository}.

    В появившееся окошко мы можем либо вставить ссылку на репозиторий,
которую мы скопировали раньше или, если вы вошли в свой аккаунт на
Github, выбрать нужный репозиторий по ссылке. Также нам нужно указать
папку в котором будет располагаться наш локальный репозиторий.

\textbf{Тут мы выбираем из списка репозиторий}

\textbf{Тут мы вставляем ссылку на репозиторий}

Вне зависимости от выбора, все файлы с удаленного репозитория перейдут в
указанную папку.

    \subsection{Добавляем и изменяем
файлы}\label{ux434ux43eux431ux430ux432ux43bux44fux435ux43c-ux438-ux438ux437ux43cux435ux43dux44fux435ux43c-ux444ux430ux439ux43bux44b}

Теперь давайте создадим в нашей папке новый текстовый документ с
сообщением Hello world!

Если мы откроем \textbf{GitHub Desktop}, мы увидим что наш файл увидела
система и пометила как добавление нового файла, отметив зеленым плюсом.
Справа отображается, что именно сделали с файлом: зеленым выделены
добавленные фрагменты.

Теперь мы готовы сделать свой первый \textbf{коммит} (commit). По факту
это фраза означает внесения изменения в текущую ветку. Чтобы это
сделать, нужно написать краткое сообщение, отражающее суть изменений,
чтобы потом было проще в них ориентироваться. В данном случае мы
добавили новый текстовый файл (сообщение может быть на любом языке,
необязательно на английском). Github сам нам подсказал название коммита.
Так же мы можем добавить описание изменений, чтобы другим пользователям
было проще.

Когда мы готовы сделать коммит, нажимаем кнопку \textbf{Commit to
master}. Это означает сделать коммит в ветку \textbf{master}, про сами
ветки расскажем чуть позже. Но мы сделали только коммит, теперь нужно
чтобы изменилсь файлы в удаленном репозитории. Для этого нажимаем кнопку
сверху \textbf{Push origin}

Если все прошло успешно, и изменения запушились в удаленный репозиторий,
то, обновив его страницу на GitHub, мы увидим новый файл hello world.txt

Поверьте, адекватные описания коммитов - это очень важно!

    Теперь давайте создадим файл на GitHub и скопируем его в локальный
репозиторий. Нажимаем кнопку \textbf{Create new file} и называем его
\textbf{newfile}.

Осталось прописать коммит и сделать его, нажав \textbf{Commit new file}

Откроем \textbf{Github Desktop} увидим что система сама определила, что
произошел внешний коммит и наши файлы нужно обновить (Если невидно
изменений нажмите F5 или перезапустите приложения). Нажимаем на
\textbf{Pull origin} и скачиваем файлы в свой локальный репозиторий.

    \subsection{Верните все
назад!}\label{ux432ux435ux440ux43dux438ux442ux435-ux432ux441ux435-ux43dux430ux437ux430ux434}

Любой коммит можно отменить, щелкнув по нему правой кнопкой мыши и
выбрав \textbf{Revert this commit}. Так, если я проведу эту процедуру со
своим последним коммитом и запушу изменения на GitHub, то newfile файл
там исчезнет. Чтобы посомтреть историю коммитов, нужно нажать на
\textbf{History}.

Откатывать коммиты можно также на самом сайте либо просматривать прошлые
версии файлов.

    \subsection{Клонирование чужих
репозиториев}\label{ux43aux43bux43eux43dux438ux440ux43eux432ux430ux43dux438ux435-ux447ux443ux436ux438ux445-ux440ux435ux43fux43eux437ux438ux442ux43eux440ux438ux435ux432}

Клонировать можно не только свои репозитории, но и чужие. Для этого
найдите нужный репозиторий в поиске на github. И выбираем \textbf{Clone
or Download}.

Далее делаем все как и при копировании своего репозитория, только в
данном случае доступен вариант клонировать только по ссылке

Что это нам дает? Это позволяет получать файлы, сразу после их
добавления или изменения и не требует захода най сайт и ручной проверки
на изменения. Этим же образом удобно клонировать репозиторий курса и
получать новые блокноты после каждого занятия.

    \subsection{Fork
репозитория}\label{fork-ux440ux435ux43fux43eux437ux438ux442ux43eux440ux438ux44f}

Fork репозитория это возможность скопировать \emph{чужой} репозитория на
свой аккаунт и вносить любые изменения в него, без изменения
оригинального репозитория. Можно сделать форк любого доступного
репозитория. При создании форка нас спросят в какой аккаунт мы хотим его
добавить.

В чем же отличие от клонирования репозитория? При клонировании мы только
\emph{используем} файлы оригинального репозитория и при создании коммита
с какими-то изменениями, Github Desktop скажет нам что у нас нет допуска
и сам предложит сделать форк (Если допуск к этому репозиторию у нас
есть, то сделать коммит мы сможем). А если мы сделали форк, то изменения
уйдут в нашу копию в нашем аккаунте

Если вы хотите править блокноты наших занятий и вносить их измения - вы
можете сделать собственный fork и это никак не повлияет на основной
репозиторий курса.

Fork может быть полезен при разработки открытого ПО, например, мы
сделали форк алгоритма сжатия, в нем мы изменили функцию сжатия и теперь
алгоримт сжимает в 10 раз лучше. Мы можем сделать \textbf{Pull request},
т.е. запросить у создателя оригинального репозитория с алгоритмом
сжатия, интегрировать наши изменения в его репозиторий. Однаком при это
этом могут возникнуть конфликтные ситуации, которые требуют ручного
разбора. В данном блоконте они не разбираются.

    \subsection{Ветки}\label{ux432ux435ux442ux43aux438}

В \textbf{git} есть понятие веток (branch), т.е. мы можем иметь
несколько независимых веток при работе. Коммит делается в конкретную
ветку, по умолчанию это ветка \textbf{master}. Создать новую ветку можно
как на сайте, так и в приложении. Для этого нужно выбрать вкладку
\textbf{Current branch} и нажать на \textbf{New branch}

Выбираем имя и в эту ветку пойдет вся информация с ветки \textbf{master}
в том числе и все файлы

И теперь мы можем переключать ветки и вносить изменения в конкретную
ветку, не затрагивая основную, в данном случае \textbf{master}.
Например, мы удалим один файл, и изменим другой. Удаленный файл будет
отмечен красным минусом, а измененный желтой точкой. При этом справа
видно что мы работаем в ветке \textbf{Features}.

Делаем коммит в новую ветку и смотрим что произошло. Как мы видим в
ветке \textbf{master} все осталось как прежде

А вот в ветке \textbf{Features} удаленного файла нет. Переключить ветки
можно, нажав на кнопку \textbf{Branch} с названием ветки

Ветки удобно использовать для добавления новых функция, что они не
ломали \emph{рабочий} код до новой функции. После разработки ветки можно
совместить сделав так называемый \textbf{Pull request}.

    \subsection{Создание репозитория из Github
Desktop}\label{ux441ux43eux437ux434ux430ux43dux438ux435-ux440ux435ux43fux43eux437ux438ux442ux43eux440ux438ux44f-ux438ux437-github-desktop}

Как говорилось ранее, новый репозиторий можно создать и из самого
приложения. Для этого идем в \textbf{File/New repository}

Указываем все данные аналогично тому как создавали на сайте и нажимаем
\textbf{Create repository}

Незабудьте нажать на \textbf{Publish repository}, чтобы он ушел на сайт.

Далее еще раз укажите имя уже на сайте и все.

    \subsection{Дополнительные
ссылки}\label{ux434ux43eux43fux43eux43bux43dux438ux442ux435ux43bux44cux43dux44bux435-ux441ux441ux44bux43bux43aux438}

Дополнительно почитать про все это можно по данным ссылкам:

\begin{itemize}
\tightlist
\item
  Про git и принципы работы с системами контроля версий (более
  техническая статья) \href{https://habr.com/ru/post/440816/}{Хабр}
\item
  Введение в работу с git
  \href{https://tproger.ru/translations/beginner-git-cheatsheet/}{TProger}
\item
  Git через консоль
  \href{https://htmlacademy.ru/blog/useful/git/git-console}{Html
  Academy}
\item
  Модель git flow \href{https://habr.com/ru/post/106912/}{Хабр}
\end{itemize}


    % Add a bibliography block to the postdoc
    
    
    
    \end{document}
